%&latex

\documentclass[style=sailor,size=14pt,display=slidesnotes]{powerdot}

\begin{document}

%+Title
\title{Data Reduction Using Splines\\ with Free\ Knots}
\author{Will Frey}
\date{\today}
\maketitle
%-Title

\begin{slide}{B-Splines}

  Power-Dot package automatically generates outline of slides.
  To take correct outline under \TeX-Word click F5 key.

  The easiest way to create overlays is in using 
  the \texttt{\textbackslash pause} command.
  Like following ...

  \begin{itemize}
    \item This is the first item\pause
    \item There is more ...
  \end{itemize}

  \texttt{\textbackslash section} command creates separate slide 
  which is shown on next frame.

\end{slide}

\section{Advanced Overlays}



\begin{slide}{Using Transparency}

  To learn more about PowerDot package 
  read original documentation.

  There is just demonstration of using transparency 
  for overlays.

  \begin{enumerate}[type=1]
    \item<1> First Item
    \item<2> Second Item
    \item<3> Third Item
  \end{enumerate}

\end{slide}

\end{document}


